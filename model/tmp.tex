\begin{figure*}  \begin{center}  \includegraphics{Figure5.png}  \caption{{\bf Quiescence stabilizes global brain dynamics in {\em C. elegans} } .
( A ) We analyze whole - brain dynamics from previous experiments in which worms were exposed to varying levels of $\text{O}_2$ concentration \cite{Nichols2017} .
We show the background subtracted fluorescence signal $\Delta F/F_0$ from 101 neurons while $\text{O}_2$ concentration changed in 6 minute periods : low $\text{O}_2$ ( 10 \% ) induces a quiescent state ; high $\text{O}_2$ ( 21 \% ) induces an active state .
( B ) We plot the distribution of maximum real eigenvalues ( $\lambda_r$ ) for the active and quiescent states .
The active state is associated with substantial unstable dynamics , while the dynamics of the quiescent state is predominately stable which is consistent with putative stable fixed point dynamics .
( C ) We plot the average maximum real eigenvalue as the $\text{O}_2$ concentration is changed .
We align the time series from different worms to the first frame of increased $O_2$ concentration and show the accompanying increase in the maximum real eigenvalue , which crosses and remains near to the instability boundary .
The shaded region corresponds to a bootstrapped 95 \% confidence interval and curves were smoothed using a 5 - frame running average . }
\label{fig:Panel5}  \end{center}  \end{figure*}